\documentclass[12pt,a4paper,twoside]{report}

\input{defs}

\begin{document}
\makeatletter

\author{Damien Deprez}
\title{FSAB 1402 - Informatique 2\\Synthèse}
\date{\today}
\titre{\@title}{\@author}{\@date}{}{}

\tableofcontents

\chapter{Introduction}
\section{Paradigmes}
Un paradigme est une approche pour programmer un ordinateur basée sur un ensemble cohérent de principe ou sur une théorie mathématique. Il met ensemble des types de données et leurs opérations, un langage pour écrire des programmes, une manière de raisonner sur les programmes.
\section{Les variables}
Une variable est constitué de deux éléments : 
\begin{itemize}
\item L'identificateur : C'est la séquence de lettre que l'on tape. Il commence toujours par une majuscule.
\item La variable en mémoire : C'est la partie de mémoire de l'ordinateur dans lequel est stocké la valeur de la variable.
\section{Fonction}
Une fonction est 
\end{itemize}
\end{document}