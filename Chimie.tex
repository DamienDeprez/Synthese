\documentclass[12pt,a4paper,twoside]{report}

\usepackage{titlesec}
\usepackage[usenames,dvipsnames]{color}
\usepackage[top=1.5cm,bottom=1.5cm,left=1cm,right=1cm]{geometry}
\usepackage[utf8]{inputenc}
\usepackage[T1]{fontenc}
\usepackage{graphicx}
\usepackage{wallpaper}
\usepackage{wrapfig}
\usepackage{hyperref}

\hypersetup{pdfborder={0 0 0}}



\titleformat{\part}
{\centering\fontfamily{pag}\fontsize{30}{30}\selectfont}
{\fontfamily{pag}\fontsize{30}{30}\selectfont Partie \ \thepart \ - }
{0pt}
{}{}

%Modifie le format des chapitres
\titleformat{\chapter}
{\it \color{Aquamarine} \fontfamily{pag}\fontsize{20.74}{20}\selectfont}
{\it \fontfamily{pag}\fontsize{20.74}{20}\selectfont Chapitre \ \thechapter \ - }
{0pt}
{}{}
\titlespacing{\chapter}{0pt}{0.5cm}{0.5cm}[0pt]

%Modifie le format des sections
\titleformat{\section}
{\color{LimeGreen}\fontfamily{pag}\fontsize{15}{15}\selectfont}
{\fontfamily{pag}\fontsize{15}{15}\selectfont \thesection }
{5pt}
{}{}
\titlespacing{\section}{1cm}{0.5cm}{0.25cm}[0cm]

%Modifie le format des sous-secions
\titleformat{\subsection}
{\color{Dandelion}\fontfamily{pag}\fontsize{12}{12}\selectfont}
{\fontfamily{pag}\fontsize{12}{12}\selectfont  \thesubsection}
{5pt}
{}{}
\titlespacing*{\subsection}{2cm}{0.1cm}{0.1cm}[0cm]

%Modifie l'espace avant les paragraphes
\setlength{\parindent}{0pt} 
\makeatletter
% Commande créeant une page de titre avec un Titre, un auteur et une date et insère une image en arrière plan au format jpg
\newcommand{\titre}[5]
{
	%\@ifmtarg{#4}{\color{#5}}{\TileWallPaper{21cm}{30cm}{#4.jpg}\color{#5}} % si le 4è arguments en non vide, insère une image et change la couleur à #5
	\begin{titlepage}
		\vspace{2cm} % insère un espace de 2 cm avant le titre
		\begin{center}\bfseries
			\fontfamily{pag}\fontsize{35}{35}\selectfont{#1} % écrit le titre en taille 40 avec une police d'Avant Garde
		\end{center}
		\begin{flushright}
				\fontfamily{pag}\fontsize{20}{20}\selectfont{#2} % écrit l'auteur sous le titre aligné à droite en taille 20 
		\end{flushright}
		\begin{center}
		\vfill % repli le reste de la page		
		\fontfamily{pag}\fontsize{20}{20}\selectfont{#3} % écrit la date en bas et au centre de la page en taille 20
		\end{center}
		\end{titlepage} % fin de la page de titre
			\color{black} % met la couleur du reste du texte à noir
	\ClearWallPaper % supprime l'image d'arrière plan
}

\begin{document}
\makeatletter

\author{Damien Deprez}
\title{FSAB 1302 - Chimie 2\\Synthèse}
\date{\today}
\titre{\@title}{\@author}{\@date}{}{}

\tableofcontents
\part{Thermodynamique}
\chapter{Gaz parfait et théorie cinétique des gaz}
\section{Théorie de Bernoulli}
\subsection{Hypothèse}
\begin{itemize}
\item Un gaz est composé de beaucoup de sphère de diamètre $d$
\item Le diamètre $d$ est petit par rapport à la distance intermoléculaire.
\item La distance intermoléculaire est petite par rapport à la taille macroscopique du système
\item Il n'y a pas d'interaction entre les molécules (attraction/répulsion).
\item Les lois de la mécanique classique sont d'applications mais pas celle de la mécanique quantique.
\item Les parois n'absorbent pas d'énergie.
\end{itemize}
Les hypothèses sont valables à faible pression(<100 bar).

La pression d'un gaz est due aux chocs des molécules sur la parois.
$$ p =\frac{1}{3} \rho\bar{c^2}$$
\section{Théorie des gaz parfait}
\subsection{Hypothèse}
\begin{itemize}
\item on suppose que la masse est constante ainsi que la température.
\end{itemize}
\subsection{Loi de Boyle}
$$pV=cst$$ 
Attention dans l'expérience de Boyle, la pression reste faible.
\subsection{Loi de Charles}
$$V(\theta)=V(0 \degres C)(1+\alpha \theta)$$

A faible pression, le coefficient $\alpha$ est le même pour tous les gaz indépendant de la température.
$$\frac{V(\theta)}{\alpha_0V(0\degres C)}=\alpha^{-1}_0 +\theta\textmd{ avec }\alpha_0^{-1}=273.15 \left[K\right]$$
$$\Rightarrow \frac{V}{T}=cst$$
\subsection{Loi de Boyle, Loi de Charles et Hypothèse d'Avogadro}
\begin{itemize}
\item Loi de Boyle : $pV=cst$
\item Loi de Charles : $V/T=cst$
\item Hypothèse d'Avogadro : $pV=nRt$
\end{itemize}
\section{Distribution de Maxwell}
\subsection{Hypothèse}
\begin{itemize}
\item Le milieu gazeux est homogène : les propriétés ne dépendent pas de la position.
\item Le milieu est isotrope : les propriétés ne dépendent pas de la direction
\item Il existe une distribution au sens statique de la vitesse (énergie cinétique)
\item Les propriétés macroscopiques sont le reflet du comportement microscopique moyen.
\end{itemize}
\subsection{Espace de Vitesse}
Soit $f=\frac{d\phi}{dc}$ la dérivée de la forme mathématique de la distribution de vitesse par rapport à la vitesse. Alors, $$\int_{-\infty}^{\infty}{\int_{-\infty}^{\infty}{\int_{-\infty}^{\infty}{f_1(c_1)f_2(c_2)f_3(c_3)dc_1dc_2dc_3}}}=1$$
Si le milieu est isotrope (invariance des propriétés physique selon la direction), alors $F(c)=f_1(c_1)f_2(c_2)f_3(c_3)$.
$$F(c)dc=f_1(c_1)f_2(c_2)f_3(c_3)dc_1dc_2dc_3=4\pi c^{2} \frac{M^{3/2}}{2\pi} e^{-\frac{M}{2RT}c^{2}}$$
Vitesse intéressantes : 
\begin{itemize}
\item vitesse la plus probable $$c_p=\sqrt{\frac{2RT}{M}}$$
\item vitesse moyenne
$$\bar{c}=\sqrt{\frac{4}{\pi}}c_p$$
\item vitesse quadratique moyenne
$$\bar{c^{2}}=\sqrt{\frac{3}{2}}c_p$$
\end{itemize}
\section{Théorie cinétique des gaz}
\subsection{Énergie cinétique d'un gaz monoatomique}
\end{document}