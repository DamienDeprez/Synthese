\documentclass[12pt,a4paper,twoside]{report}

\input{defs}

\begin{document}
\makeatletter

\author{Damien Deprez}
\title{FSAB 1302 - Chimie 2\\Synthèse}
\date{\today}
\titre{\@title}{\@author}{\@date}{}{}

\tableofcontents
\part{Thermodynamique}
\chapter{Gaz parfait et théorie cinétique des gaz}
\section{Théorie de Bernoulli}
\subsection{Hypothèse}
\begin{itemize}
\item Un gaz est composé de beaucoup de sphère de diamètre $d$
\item Le diamètre $d$ est petit par rapport à la distance intermoléculaire.
\item La distance intermoléculaire est petite par rapport à la taille macroscopique du système
\item Il n'y a pas d'interaction entre les molécules (attraction/répulsion).
\item Les lois de la mécanique classique sont d'applications mais pas celle de la mécanique quantique.
\item Les parois n'absorbent pas d'énergie.
\end{itemize}
Les hypothèses sont valables à faible pression(<100 bar).

La pression d'un gaz est due aux chocs des molécules sur la parois.
$$ p =\frac{1}{3} \rho\bar{c^2}$$
\section{Théorie des gaz parfait}
\subsection{Hypothèse}
\begin{itemize}
\item on suppose que la masse est constante ainsi que la température.
\end{itemize}
\subsection{Loi de Boyle}
$$pV=cst$$ 
Attention dans l'expérience de Boyle, la pression reste faible.
\subsection{Loi de Charles}
$$V(\theta)=V(0 \degres C)(1+\alpha \theta)$$

A faible pression, le coefficient $\alpha$ est le même pour tous les gaz indépendant de la température.
$$\frac{V(\theta)}{\alpha_0V(0\degres C)}=\alpha^{-1}_0 +\theta\textmd{ avec }\alpha_0^{-1}=273.15 \left[K\right]$$
$$\Rightarrow \frac{V}{T}=cst$$
\subsection{Loi de Boyle, Loi de Charles et Hypothèse d'Avogadro}
\begin{itemize}
\item Loi de Boyle : $pV=cst$
\item Loi de Charles : $V/T=cst$
\item Hypothèse d'Avogadro : $pV=nRt$
\end{itemize}
\section{Distribution de Maxwell}
\subsection{Hypothèse}
\begin{itemize}
\item Le milieu gazeux est homogène : les propriétés ne dépendent pas de la position.
\item Le milieu est isotrope : les propriétés ne dépendent pas de la direction
\item Il existe une distribution au sens statique de la vitesse (énergie cinétique)
\item Les propriétés macroscopiques sont le reflet du comportement microscopique moyen.
\end{itemize}
\subsection{Espace de Vitesse}
Soit $f=\frac{d\phi}{dc}$ la dérivée de la forme mathématique de la distribution de vitesse par rapport à la vitesse. Alors, $$\int_{-\infty}^{\infty}{\int_{-\infty}^{\infty}{\int_{-\infty}^{\infty}{f_1(c_1)f_2(c_2)f_3(c_3)dc_1dc_2dc_3}}}=1$$
Si le milieu est isotrope (invariance des propriétés physique selon la direction), alors $F(c)=f_1(c_1)f_2(c_2)f_3(c_3)$.
$$F(c)dc=f_1(c_1)f_2(c_2)f_3(c_3)dc_1dc_2dc_3=4\pi c^{2} \frac{M^{3/2}}{2\pi} e^{-\frac{M}{2RT}c^{2}}$$
Vitesse intéressantes : 
\begin{itemize}
\item vitesse la plus probable $$c_p=\sqrt{\frac{2RT}{M}}$$
\item vitesse moyenne
$$\bar{c}=\sqrt{\frac{4}{\pi}}c_p$$
\item vitesse quadratique moyenne
$$\bar{c^{2}}=\sqrt{\frac{3}{2}}c_p$$
\end{itemize}
\subsection{Illustration : l'effusion d'un gaz}
L'effusion d'un gaz se fait sans choc entre molécules près de la paroi et sans choc avec la paroi. C'est différent de la diffusion qui implique des chocs.
Le débit de molécules qui passent est donné par le taux de collision avec la surface de passage.
$$\frac{dN}{dt}=\frac{N}{V}A\int_{0}^{\infty}çc_xf(c_x)dc_x=A\frac{p}{(2\pi k_BTm)^{1/2}}$$
\section{Théorie cinétique des gaz}
\subsection{Énergie cinétique d'un gaz monoatomique}
L'énergie cinétique d'un gaz monoatomique est prédite correctement.
$$\frac{m\bar{c^2}}{2}=\frac{3}{2}k_bT$$
Pour une mole, l'énergie cinétique devient
$$U_m=N_A\frac{m\bar{c^2}}{2}=\frac{3}{2}RT$$
Comme les molécules monoatomiques n'ont que trois degrés de liberté, chaque degré stocke un tiers de l'énergie cinétique.

\subsection{Énergie cinétique d'un gaz diatomique}
Dans le cas des gaz diatomique, l'énergie cinétique n'est pas uniquement stockée dans les degrés de liberté de translation mais aussi dans les degrés de libertés de rotation ainsi que ceux de vibration qui compte double (énergie cinétique + potentielle).
Pour faciliter les calculs, en fonction de la gamme de température, on peut diviser l'énergie cinétique en trois catégorie:
\begin{itemize}
 \item Basse température (<100 K) $3/2 RT$
 \item Température normale (<1000 K) $5/2 RT$
 \item Haute température (>1000 K) $7/2 RT$
\end{itemize}

\end{document}